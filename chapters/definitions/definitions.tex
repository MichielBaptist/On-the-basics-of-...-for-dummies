\chapter*{Definitions and notation}

In this section I present an overview of the mathematical notation I employ and some of the more common definitions. Whenever a definition is mentioned in the text, this chapter is referred.

\section*{Notation}

\section*{Commonly used definitions}

Humans usually like to compare things to each other. Which apple is riper, juicier and generally better looking? Do I procrastinate, do I try to be productive? Often one or several criteria are compared. Color of the apple, size of the apple, and so on. In mathematics, humans have implemented a very similar notion upon objects such as vectors and matrices. This notion is called a \textit{norm}. It's definition is rather simple, but elegant:

\begin{defn} [Mathematical norm]
Given a vector space $V$ in over a field in $\Real$ or $\Compl$. A norm $\norm{.}$ is a function $\norm{.}:V \mapsto \Real$ such that $\forall v, u \in V$:
\begin{enumerate}
\item $\norm{v} \geq 0$
\item $\norm{v + u} \leq \norm{v} + \norm{u}$
\item $\norm{av} = |a|\norm{v}$ $\forall a \in \Real$
\item $\norm{v} = 0 \Leftrightarrow v = 0$
\end{enumerate}
Condition 2 is called the triangle inequality. Note that the definition of norm might allow many definitions. Not only the norms such as the euclidean norm (p-norm) for example.
\end{defn}

\begin{defn}[The $\mathcal{L}_p$ norm or the p-norm]
The $\mathcal{L}_p$ norm or p-norm $\norm{.}_p$ is defined over $n$ dimensional vector spaces. It is given by:
\begin{equation}
\norm{x}_p = \bigg( \sum_{i=1}^{n} |x_i|^p \bigg)^\frac{1}{p}
\end{equation}
\end{defn}
These types of norms are most commonly used in statistics and mathematics.
\begin{defn}[Frobenius norm]
The frobenius norm is defined over $m\times n$ matrices $X \in \Real^{m \times n}$. It is the natural extension of the 2-norm for vectors. It is defined as: 
\begin{equation}
\begin{split}
\norm{X}_F &= \sqrt{\sum_{i=1}^{m}\sum_{j=1}^{n}|x_{ij}|^2}\\
&= \sqrt{trace(X^TX)} \\
\end{split}
\end{equation}
\end{defn}